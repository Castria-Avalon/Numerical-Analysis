\documentclass[a4paper]{article}
\usepackage[affil-it]{authblk}
\usepackage{amsmath}
\usepackage{amsfonts}
\usepackage{forest}
\usepackage{graphicx}
\usepackage{caption}
\usepackage{listings}
\usepackage[backend=bibtex,style=numeric]{biblatex}


\usepackage{geometry}
\geometry{margin=1.5cm, vmargin={0pt,1cm}}
\setlength{\topmargin}{-1cm}
\setlength{\paperheight}{29.7cm}
\setlength{\textheight}{25.3cm}

\addbibresource{citation.bib}

\begin{document}
\forestset{
  folder/.style={
    for tree={
      grow=east,
      edge={->},
      l sep=15pt,
      if n children=0{tier=word}{}
    },
    delay={where content={}{shape=folder,draw,minimum height=3ex}{}, 
           where content={}{tier=parent}{}, 
           if n children=0{}}
  }
}
\nocite{*}
% =================================================
\title{Numerical Analysis Project Design Report}

\author{ Wu Jinhan 3220102328
  \thanks{Electronic address: \texttt{3220102328@zju.edu.cn}}}
\affil{(Information and Computing Science 2201), Zhejiang University }


\date{Due time: \today}

\maketitle



% ============================================
\section*{A. Programming Design}

\begin{forest}
  for tree={
    grow=east, 
    edge={->}, 
    parent anchor=east,
    child anchor=west,
    align=center,
    l sep+=10pt, 
    anchor=west
  }
  [Spline
    [
        [class SplineConfigParams
            [JSONload]
        ],
        [class ppSpline],
        [class BSpline]
    ]
  ]
\end{forest}

\subsection*{class SplineConfigParams}
The main implementation focuses on encapsulating spline parameters, with the core consisting of two constructor functions:
\begin{enumerate}
    \item Manual Construction of Spline Parameters: One constructor allows the user to manually define the spline parameters.
    \item  JSON-Based Construction: The other constructor leverages `jsonspp` to read parameters from a JSON file, enabling the control of spline parameters through external configuration.
\end{enumerate}
This dual approach ensures flexibility in defining and managing spline parameters.
This approach ensures an efficient and systematic way to calculate spline values.

\subsection*{class ppSpline \& BSpline}
The main implementation methods of the spline class are all realized through these two classes. The core idea is as follows:
\begin{enumerate}
    \item Spline Parameter Introduction: By introducing spline parameters, the process begins with constructing the parameter matrix. 
    \item Parameter Matrix Construction and Solving: The parameter matrix is self-constructed and solved to determine the coefficients.
    \item Spline Value Calculation: When retrieving the value of the spline, back-substitution is performed using the obtained coefficients.
\end{enumerate}

\subsection{Auxiliary Class: PlotRW}
The main functionality of this class is to export data from a vector and store the fitted data points in a CSV file. This allows the data to be easily read and processed by Python for further analysis or visualization.


\section*{B. Mathematical Proof}

\subsection*{PP-form Spline}

\subsubsection*{I. Cubic Polynomial Expressions}

We use the third derivative of \( S(x) \), denoted as \( S''(x) = M_j \) (\( j = 0, 1, \dots, n \)), to represent \( S(x) \). Since \( S(x) \) is a continuous function in the interval \([x_j, x_{j+1}]\), it can be expressed as:
\[
S''(x) = M_j \frac{(x_{j+1} - x)^3}{6h_j} + M_{j} \frac{(x - x_j)^3}{6h_j}
\]
For the second integral and the use of \( S(x_j) = y_j \) and \( S(x_{j+1}) = y_{j+1} \), the following cubic polynomial expressions are obtained:
\[
S(x) = M_j \frac{(x_{j+1} - x)^3}{6h_j} + M_{j+1} \frac{(x - x_j)^3}{6h_j} + \left( y_{j} - \frac{M_j h_j^2}{6} \right) \frac{x_{j+1} - x}{h_j} + \left( y_{j+1} - \frac{M_{j+1} h_j^2}{6} \right) \frac{{x} - x_j}{h_j}, \quad j = 0, 1, \dots, n-1
\]
Here, \( M_j \) (\( j = 0, 1, \dots, n \)) is unknown. But based on the text book, we have following conclusion to determine \( M_j \):
\[
\mu_j M_{j-1} + 2 M_j + \lambda_j M_{j+1} = d_j, \quad j = 1, 2, \dots, n-1,
\]
Where:
\begin{align*}
&\mu_j = \frac{h_{j-1}}{h_{j-1} + h_j}, \quad \lambda_j = \frac{h_j}{h_{j-1} + h_j},\\
&d_j = 6 \frac{f[x_j, x_{j+1}] - f[x_{j-1}, x_j]}{h_{j-1}+h_j}  = 6 f[x_{j-1}, x_j, x_{j+1}], \quad j = 1, 2, \dots, n-1,
\end{align*}

\subsubsection*{II. The construction of the matrix corresponding to the boundary conditions}

For the complete boundary condition, two equations are derived:
\begin{align*}
    &2 M_0 + M_1 = \frac{6}{h_0} \left( f[x_0, x_1] - f'_0 \right),\\
    &M_{n-1} + 2 M_n = \frac{6}{h_{n-1}} \left( f'_n - f[x_{n-1}, x_n] \right).
\end{align*}
If we set \( \lambda_0 = 1, d_0 = \frac{6}{h_0} \left( f[x_0, x_1] - f'_0 \right), \mu_0 = 1, d_n = \frac{6}{h_{n-1}} \left( f'_n - f[x_{n-1}, x_n] \right) \), then equations can be written in matrix form:
\[
\begin{pmatrix}
    2&\lambda_0 \\
    \mu_1&2&\lambda_1 \\
     &\ddots&\ddots&\ddots\\
      & & \mu_{n-1}&2&\lambda_{n-1}\\
      & & & \mu_n&2
\end{pmatrix}
\begin{pmatrix}
    M_0\\
    M_1\\
    \vdots\\
    M_{n-1}\\
    M_n
\end{pmatrix}
=
\begin{pmatrix}
    d_0\\
    d_1\\
    \vdots\\
    d_{n-1}\\
    d_n
\end{pmatrix}
\]

For the natural boundary condition, the end point equations are directly obtained:
\[
M_0 = f''_0, \quad M_n = f''_n. 
\]
If we set \( \lambda_0 = \mu_0 = 0, d_0 = 2f''_0, d_n = 2f''_n \), then equations can be written in matrix form like above.

For the periodic boundary condition, we obtain:
\[
M_0 = M_n, \quad \lambda_n M_1 + \mu_n M_{n-1} + 2 M_n = d_n, 
\]
Where:
\begin{align*}
    &\lambda_n = \frac{h_0}{h_{n-1} + h_0}, \quad \mu_n = 1 - \lambda_n,\\
    &d_n = \frac{6( f[x_0, x_1] - f[x_{n-1}, x_n] )}{h_0 + h_{n-1}} .
\end{align*}
Then we get the matrix form
\[
\begin{pmatrix}
    2&\lambda_1& & & \mu_1 \\
    \mu_2&2&\lambda_2 \\
     &\ddots&\ddots&\ddots\\
      & & \mu_{n-1}&2&\lambda_{n-1}\\
     \lambda_n & & & \mu_n&2
\end{pmatrix}
\begin{pmatrix}
    M_1\\
    M_2\\
    \vdots\\
    M_{n-1}\\
    M_n
\end{pmatrix}
=
\begin{pmatrix}
    d_1\\
    d_2\\
    \vdots\\
    d_{n-1}\\
    d_n
\end{pmatrix}
\]

\subsection*{B-Spline(3 degrees)}
\subsubsection*{I. Node construction}

Based on the textbook, we have:
\[
S_n^{n-1}(t)=\sum^{N-2}_{i=1}a_iB_i^n(t),
\]
and the value of each S node is determined by the following basis splines:
\begin{align*}
    S(t_0)&\quad\Rightarrow\quad B^n_{-n},\ldots\, ,B_{-1}^n,\\
    S(t_1)&\quad\Rightarrow\quad B^n_{1-n},\ldots\, ,B_{0}^n,\\
    &\vdots\\
    S(t_{N-1})&\quad\Rightarrow\quad B^n_{N-n-1},\ldots\, B_{N-2}^n,\\
\end{align*}

To this end, we extend the nodes: the nodes in the forward extension are decreased by 1 incrementally, while the nodes in the backward extension are increased by 1 incrementally, in order to maintain the monotonic increase of the nodes.

\subsubsection*{II. The boundary conditions}

First, we have the differentiation formula for the B-spline basis functions of order \( B \)
\[
\frac{d}{dx}B_i^n(x)=\frac{nB_i^{n-1(x)}}{t_{i+n}-t_i}-\frac{nB_{i+1}^{n-1(x)}}{t_{i+n+1}-t_{i+1}}.
\]
Subsequently, we can derive the first-order and second-order derivatives corresponding to the boundary conditions
\begin{align*}
    &S''(t_0)=\frac{6}{t_1-t_{-1}}\left( \frac{a_{-3}-a_{-2}}{t_1-t_{-2}}- \frac{a_{-2}-a_{-1}}{t_2-t_{-1}}\right)B^1_{-1}(t_0),\\
    &S''(t_{N-1})=\frac{6}{t_N-t_{N-2}}\left( \frac{a_{N-4}-a_{N-3}}{t_N-t_{N-3}}- \frac{a_{N-2}-a_{N-3}}{t_{N+1}-t_{N-2}}\right)B^1_{N-2}(t_{N-1}),\\
    &S'(t_0)=\frac{3(-a_{-3}+a_{-2})}{t_1-t_{-2}}B^2_{-2}(t_0)-\frac{3(a_{-2}-a_{-1})}{t_2-t_{-1}}B^2_{-1}(t_0),\\
    &S'(t_{N-1})=\frac{-3(a_{N-4}+a_{N-3})}{t_N-t_{N-3}}B^2_{N-3}(t_{N-1})-\frac{3(a_{N-2}-a_{N-3})}{t_{N+1}-t_{N-2}}B^2_{N-2}(t_{N-1}).
\end{align*}
Subsequently, based on the definitions of the three types of periodic conditions, the solution matrix for the coefficients \(a_i\) can be easily obtained.


% ===============================================
\subsection*{ \center{\normalsize {Acknowledgement}} }
\printbibliography

\end{document}